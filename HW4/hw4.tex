\documentclass{article}
\usepackage[stdmargin, noindent]{../rajeev}

\begin{document} 
\setcounter{problem}{10}
\problem{}

\subproblema{}
We first find the probability of $A_s$.
Since the ace of spades is only 1 card,
$\PP{A_s} = \frac{1}{52}$.

On the other hand,
$A_s \cap B$ is the event that both cards drawn are spades and one of them is the ace of spades.
This means the other card has 3 other options available for its suite.
The ace of spades can either be the first or second,
and adding together the respectively probabilities yields
$$\PP{A_s \cap B} = \frac{1}{52} \cdot \frac{3}{51} + \frac{3}{52} \cdot \frac{1}{51} = \frac{6}{51 \cdot 52}$$
Therefore, $\boxed{\PP{B \vert A_s} = \frac{2}{17}}$.

\subproblema{}
The probability of at least one ace being chosen is the complement of no aces being chosen.
That probability is $\PP{A^c} = \frac{48}{52} \cdot \frac{47}{51} = \frac{188}{221}$.
Therefore, $\PP{A} = 1 - \frac{188}{221} = \frac{33}{221}$.

The event $A \cap B$ implies that both cards are aces.
Therefore, $$ \PP{A \cap B} = \frac{4}{52} \cdot \frac{3}{51} = \frac{1}{13 \cdot 17} $$
Finally, $\boxed{\PP{B \vert A} = \frac{1}{33}}$.


\setcounter{problem}{23}
\problem{}
\subproblema{}
We can use the law of total probability to solve this problem.
However,
we must first define some notation for convenience.
Let $W$ be the event that the ball selected from urn II is white.
Let $X_W, X_R$ be the events that the ball transferred was white or red,
respectively.
Using the law of total probability,
we have
\begin{align*}
  \PP{W} &= \PP{W \vert X_W} \PP{X_W} + \PP{W \vert X_R} \PP{X_R} \\
         &= \frac{2}{3} \cdot \frac{1}{3} + \frac{1}{3} \frac{2}{3} \\
         &= \frac{4}{9} \\
\end{align*}
Therefore, $\boxed{\PP{W} = \frac{4}{9}}$.

\subproblema{}
Urn II will then have 2 white balls and 1 red ball.
Using the notation from the last subproblem,
$\boxed{\PP{W \vert X_W} = \frac{2}{3}}$.

\setcounter{problem}{30}
\problem{}
We simply count the probability that each ball hasn't been touched,
making sure to reduce the denominator by 1 each time.
This implies the probability to be $\frac{12}{15} \cdot \frac{11}{14} \cdot \frac{10}{12} = \boxed{\frac{11}{21}}$.

\problem{}
Let $X_1$ be the event that the first box is selected.
Let $X_2$ be the event that the second box is selected.
Let $B$ be the event that the selected marble is black.
Let $W$ be the event that the selected marble is white.
We use to law of total probability to have
\begin{align*}
  \PP{B} &= \PP{B \vert X_1} \PP{X_1} + \PP{B \vert X_2} \PP{X_2} \\
         &= \frac{1}{2} \cdot \frac{1}{2} + \frac{2}{3} \cdot \frac{1}{2} \\
         &= \boxed{\frac{7}{12}} \\
\end{align*}

For the next part of the problem, we use Bayes' theorem.
\begin{align*}
  \PP{X_1 \vert W} &= \frac{\PP{W \vert X_1}}{\PP{X_1}}  \PP{W} \\
                   &= \frac{\frac{1}{2}}{\frac{1}{2}} \pars{\PP{W \vert X_1} \PP{X_1} + \PP{W \vert X_2} \PP{X_2}} \\
                   &= \frac{1}{2} \cdot \frac{1}{2} + \frac{1}{3} \cdot \frac{1}{2} \\
                   &= \boxed{\frac{5}{12}} \\
\end{align*}




\end{document}