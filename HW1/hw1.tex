\documentclass{article}
\usepackage[stdmargin]{../rajeev}

\begin{document}
\setcounter{problem}{4} 
\problem{}

There are $\pars{9-2}+1 = 8$ possible options for the first number, $2$ possible options for the second number, and $\pars{9-1}+1 = 9$ possible options for the third number.
Multiplying these, we have $8 \cdot 2 \cdot 9 = \boxed{144}$ total options.

If the first number is a $4$, then there is only $1$ option for the first number.
The number of options for the second and third numbers do not change.
Therefore, there are $1 \cdot 2 \cdot 9 = \boxed{18}$ total options.



\setcounter{problem}{7}
\problem{}
\subproblema{}

There are no repeats, so there are $5$ choices for the first letter, $4$ choices for the second letter, $3$ choices for the third letter, $2$ choices for the fourth letter, and $1$ choice for the fifth letter.
In total, that means there are $5 \cdot 4 \cdot 3 \cdot 2 \cdot 1 = 5! = \boxed{120}$ different letter arrangements.

\vspace{-4\parskip}
\subproblema{}
We note that the P and O each appear twice, and the R, S, and E each appear once.
We assign the doubly-appearing letters first, starting with the P's.
Of the $7$ letters in each arrangement, $2$ must be occupied by P's, so there are
$\binom{7}{2}$ possible options for this.
There are $7-2=5$ letters left, so there must be $\binom{5}{2}$ options to choose where to put the O's.
There are then $3$ letters left in the arrangement, so there are $3$ options to choose where to put the R, $2$ options to choose where to put the S, and $1$ option to choose where to put the E.
Multiplying everything out,
\begin{align*}
  \binom{7}{2} \cdot \binom{5}{2} \cdot 3 \cdot 2 \cdot 1 &= 21 \cdot 10 \cdot 3 \cdot 2 \cdot 1 \\
                                                          &= \boxed{1260} \\
\end{align*}

\setcounter{problem}{10}
\vspace{-4\parskip}
\problem{}
\subproblema{}
If the relative order doesn't matter, then each book can be treated arbitrarily, regardless of its type.
There are $6$ books in total, so there are $6! = \boxed{720}$ different ways to order the books.

\vspace{-4\parskip}
\subproblema{}
Since there is only $1$ chemistry book, and the $3$ novels and $2$ math books must be together, we can instead organize objects we define to be collections.
Let the first collection be the $1$ chemistry book, let the second collection be the two math books, and let the third collection be the $3$ novels.
Within, each collection, there is only $1$ way to organize the chemistry books, $3! = 6$ ways to organize the novels, and $2! = 2$ ways to organize the math books.
We can then organize the collections.
There are 3 of them, so there must be $3! \cdot  = 6$ ways to arrange the collections.
Therefore, the total number of ways to organize the books is $3! \cdot 2! \cdot 3! = \boxed{72}$.

\vspace{-4\parskip}
\subproblema{}
Similar to the previous subproblem, the $3$ novels can be treated as one collection.
The number of ways to organize the novels within the collection is $3! = 6$.
On a slightly broader level, there are the collection of novels, the $2$ math books, and $1$ chemistry book.
There are $4$ objects in total, so the number of ways to organize them is $4! = 24$.
The total number of ways to organize the books in this way is therefore $3! \cdot 4! = \boxed{144}$.

\setcounter{problem}{25}
\problem{}
\subproblema{}

We use the binomial theorem.
\begin{align*}
  3^n &= \pars{2 + 1}^n \\
      &= \sum \limits_{0}^{k} \binom{n}{k} \pars{2}^n \pars{1}^{n-k} \\
      &= \sum \limits_{0}^{k} \binom{n}{k} \pars{2}^n \\
\end{align*}

Here, we have used the fact that $1^x = 1$ for all natural $x$.

\vspace{-4\parskip}
\subproblema{}

Using the fact that $1^{y} = 1$ for all natural $y$, we can write

\begin{align*}
  \sum \limits_{n=0}^{k} \binom{n}{k} x^k &= \sum \limits_{n=0}^{k} \binom{n}{k} x^k 1^{n-k} \\
                                          &= \boxed{\pars{x+1}^k} \\
\end{align*}

Here, we have used the binomial theorem.

\end{document}