\documentclass{article}
\usepackage[stdmargin, noindent]{../rajeev}

\begin{document}

\begin{center}
  \Large HW 7
\end{center}


\setcounter{problem}{38}
\problem{}

We must first solve for $\EE{X^2}$.
\begin{align*}
  \EE{X^2} - \EE{X}^2 &= \var{X} \\
  \EE{X^2} - 1 &= 5 \\
  \EE{X^2} &= 6 \\
\end{align*}

\subproblema{}
We use linearity of expectation, finding
\begin{align*}
  \EE{\pars{2+X}^2} &= \EE{4 + 4X + X^2} \\
                    &= 4 + 4 \EE{X} + \EE{X^2} \\
                    &= \boxed{14} \\
\end{align*}

\subproblema{}
We use the formula for variance and then linearity of expectation, finding
\begin{align*}
  \var{4+3X} &= \EE{\pars{4+3X}^2} - \EE{4+3X}^2 \\
             &= \EE{16 + 24 X + 9 X^2} - \pars{4 + 3}^2 \\
             &= 16 + 24 + 54 - 49 \\
             &= \boxed{45} \\
\end{align*}

\problem{}
Let $X$ be a binomially-distributed random variable,
that tells us the number of white balls drawn from the urn.
Then,
\begin{align*}
  \PP{X = 2} &= \binom{4}{2} \pars{\frac{1}{2}}^2 \pars{\frac{1}{2}}^2 \\
             &= \frac{6}{16} \\
             &= \boxed{\frac{3}{8}} \\
\end{align*}


\setcounter{problem}{59}
\problem{}
Let $X$ be a random variable that represents the number of accidents that happen on a highway today.

\subproblema{}
We could directly compute this using an infinite series,
but will instead use the principle of complements,
finding
\begin{align*}
  \PP{X \geq 3} &= 1 - \PP{X < 3} \\
                &= 1 - \PP{X = 0} - \PP{X = 1} - \PP{X = 2} \\
                &= 1 - e^{-3} - 3 e^{-3} - \frac{9}{2} e^{-3} \\
                &= \boxed{1 - \frac{17}{2} e^{-3}} \\
                &\approx 0.577 \\
\end{align*}

\subproblema{}
Here, $X$ is strictly greater than $0$.
Therefore,
\begin{align*}
  \PP{X \geq 3 \vert X \geq 1} &= \frac{\PP{X \geq 3}}{\PP{X \geq 1}} \\
                               &= \frac{1 - \frac{17}{2} e^{-3}}{1 - \PP{X = 0}} \\
                               &= \frac{1 - \frac{17}{2} e^{-3}}{1 - 3 e^{-3}} \\
                               &= \boxed{\frac{2e^3 - 17}{2e^3 - 6}} \\
                               &\approx 0.678 \\
\end{align*}


\setcounter{problem}{62}
\problem{}
Let $C$ be a Poisson random variable representing the number of colds the individual has in a year.
Let $R$ be the event that the indvidual is in the population for which $\lambda = 3$.
$R^c$ is then the event that the individual is in the population for which $\lambda = 5$.
We can then compute,
using Bayes' theorem and then the law of total probability,
\begin{align*}
  \PP{R \vert C = 2} &= \PP{C = 2 \vert R} \frac{\PP{R}}{\PP{C = 2}} \\
                     &= \PP{C = 2 \vert R} \frac{\PP{R}}{\PP{C = 2 \vert R} \PP{R} + \PP{C = 2 \vert R^c} \PP{R^c}} \\
                     &= \pars{\frac{9 e^{-3}}{2}} \frac{\frac{3}{4}}{\frac{3}{4} \cdot \frac{9 e^{-3}}{2} + \frac{1}{4} \cdot \frac{25 e^{-5}}{2} } \\
                     &= \pars{\frac{9 e^{-3}}{2}} \frac{\frac{3}{4}}{\frac{27e^{-3}}{8} + \frac{25e^{-5}}{8}} \\
                     &= \frac{\frac{27e^{-3}}{8}}{\frac{27e^{-3}}{8} + \frac{25e^{-5}}{8}} \\
                     &= \boxed{\frac{27e^{2}}{27 e^{2} + 25}} \\
                     &\approx 0.889
\end{align*}



\end{document}