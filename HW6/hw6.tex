\documentclass{article}
\usepackage[stdmargin, noindent]{../rajeev}

\begin{document}

\problem{}

Let $X$ be a random variable denoting the winnings.
Since doing otherwise would be cumbersome,
we make a table illustrating what the possible values of $X$ are,
as well as what ball combinations they correspond to.

\begin{center}
  \begin{tabular}{ |c|c|c| } 
    \hline
    \textbf{Value of $X$} & \textbf{Ball 1} & \textbf{Ball 2} \\ 
    \hline
    4 & Black & Black \\
    2 & Black & Orange \\
    1 & Black & White \\
    0 & Orange & Orange \\
    -1 & White & Orange \\
    -2 & White & White \\
    \hline
  \end{tabular}
\end{center}

Note that \textbf{Ball 1} and \textbf{Ball 2} may be freely switched.

This means,
\begin{align*}
  \PP{X = 4} &= \frac{4}{14} \cdot \frac{3}{13} \\
             &= \boxed{\frac{6}{91}} \\
  \PP{X = 2} &= 2 \cdot \frac{4}{14} \cdot \frac{2}{13} \\
             &= \boxed{\frac{8}{91}} \\
  \PP{X = 1} &= 2 \cdot \frac{4}{14} \cdot \frac{8}{13} \\
             &= \boxed{\frac{32}{91}} \\
  \PP{X = 0} &= \frac{2}{14} \cdot \frac{1}{13} \\
             &= \boxed{\frac{1}{91}} \\
  \PP{X = -1} &= 2 \cdot \frac{8}{14} \cdot \frac{2}{13} \\
             &= \boxed{\frac{16}{91}} \\
  \PP{X = -2} &= \frac{8}{14} \frac{7}{13} \\
             &= \boxed{\frac{4}{13}} \\
\end{align*}

\setcounter{problem}{5}
\problem{}
There are $4$ possible values of $X$: $\set{3, 1, -1, -3}$.
The first and last values correspond to all heads and all tails,
respectively.
The second and third values correspond to $2$ heads and $2$ tails,
respectively.

\begin{align*}
  \PP{X = 3} &= \pars{\frac{1}{2}}^3 \\
             &= \boxed{\frac{1}{8}} \\
  \PP{X = 1} &= \binom{3}{2} \pars{\frac{1}{2}}^2 \pars{\frac{1}{2}} \\
             &= \boxed{\frac{3}{8}} \\
  \PP{X = -1} &= \binom{3}{2} \pars{\frac{1}{2}}^2 \pars{\frac{1}{2}} \\
             &= \boxed{\frac{3}{8}} \\
   \PP{X = -3} &= \pars{\frac{1}{2}}^3 \\
             &= \boxed{\frac{1}{8}} \\
\end{align*}


\setcounter{problem}{20}
\problem{}
\subproblema{}

The only possible values of $X$ that are possible are contained in $\set{1, -1, 3}$.
When $X = 1$,
this means that you won the first game,
or that you lost the first and won the second and third.
When $X = -1$,
this means that you only won one of the last two games,
and lost the first one.
When $X = -3$,
this means that you lost all three games.

Since the only positive value attainable by $X$ is $1$,
we only need to compute $\PP{X = 1}$ to compute $\PP{X > 0}$.
However,
we will still compute $\PP{X = -1}$ and $\PP{X = -3}$,
since it is necessary information for part (c) of this question.

For the sake of notation,
we let $G_i$ be the event that the $i$th game was won for $i \in \brak{1, 3} \cap \ZZ$,
so $G_i^c$ is the event that the $i$th game was lost.
We know that the games are independent from each other,
and that $\PP{G_i} = \frac{9}{19}$ and $\PP{G_i^c} = \frac{10}{19}$.

We then have
\begin{align*}
  \PP{X = 1} &= \PP{G_1} + \PP{G_1^c \cup G_2 \cup G_3} \\
             &= \frac{9}{19} + \frac{10}{19} \cdot \pars{\frac{9}{19}}^2 \\
             &= \frac{9}{19} + \frac{10}{19} \cdot \frac{81}{361} \\
             &= \frac{9}{19} + \frac{810}{6859} \\
             &= \frac{4059}{6859} \\
  \PP{X = -1} &= \PP{G_1^c \cup G_2 \cup G_3^c} + \PP{G_1^c \cup G_2^c \cup G_3} \\
             &= 2 \cdot \pars{\frac{10}{19}}^2 \cdot \frac{9}{19} \\
             &= \frac{1800}{6859} \\
  \PP{X = -3} &= \PP{G_1^c \cup G_2^c \cup G_3^c} \\
             &= \pars{\frac{10}{19}}^3 \\
             &= \frac{1000}{6859} \\
\end{align*}

Since $X = 1$ is the only possible value of $X$ such that $X > 0$,

$$
\PP{X > 0} = \PP{X = 1} = \boxed{\frac{4059}{6859}}
$$

We note that $\frac{4059}{6859} \approx 0.592 > \frac{1}{2}$.


\subproblema{}
It highly depends on the answer to part (c).
A strategy is only a winning one if its expected value is positive, which may or may not be true.
Although each individual probability is a component of the expected value, we can't say without first calculating the expected value.

In part (c), we will show that $\EE{x} < 0$, so this isn't a winning strategy.

\subproblema{}

\begin{align*}
  \EE{X} &= 1 \cdot \PP{X = 1} - 1 \PP{X = -1} - 3 \PP{X = -3} \\
         &= \frac{1}{6859} \pars{4059 - 1800 - 3 \cdot 1000} \\
         &= \boxed{- \frac{39}{361}} \\
         & \approx - 0.108
\end{align*}

In context, this means that we will lose about $11$ cents each time this strategy is used.
Since $\EE{X} < 0$, this strategy is a losing strategy, on average.

\setcounter{problem}{24}
\problem{}
\subproblema{}
When $X=1$, only one of the coins may be heads.
Therefore,
\begin{align*}
  \PP{X = 1} &= \frac{3}{5} \cdot \frac{3}{10} + \frac{2}{5} \cdot \frac{7}{10} \\
             &= \boxed{\frac{23}{50}} \\
\end{align*}

\subproblema{}
We must first calculate $\PP{X = 0}$ and $\PP{X = 2}$.
\begin{align*}
  \PP{X = 0} &= \frac{2}{5} \cdot \frac{3}{10} \\
             &= \frac{3}{25} \\
  \PP{X = 2} &= \frac{3}{5} \cdot \frac{7}{10} \\
             &= \frac{21}{50} \\
\end{align*}

Now, we can calculate $\EE{X}$.
\begin{align*}
  \EE{X} &= 0 \cdot \frac{3}{25} + 1 \cdot \frac{23}{50} + 2 \cdot \frac{21}{50} \\
         &= \frac{23}{50} + \frac{42}{50} \\
         &= \boxed{\frac{13}{10}} \\
         &= 1.3 \\
\end{align*}


\end{document}