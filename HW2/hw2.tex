\documentclass{article}
\usepackage[stdmargin, noindent]{../rajeev}

\begin{document}
\setcounter{problem}{2} 
\problem{}

$EF$ is the event that the sum of the first roll is odd and in second roll, at least one die lands on 1.
$E \cup F$ is the event that the sum of a roll is odd and one die lands on 1.
$FG$ is the event that the first roll has at least 1 die that lands on 1 and the second roll sums to 5.
$EF^c$ is the event that the first roll sums to an odd number and the none of the dice in the second roll land on 1.
$EFG$ is the event that the first roll sums to an odd number, the second roll has at least one die that lands on 1, and the third roll sums to 5.

\setcounter{problem}{8}

\subproblema{}
The events are mutually exclusive, so we have $\PP{A \cup B} = \PP{A} + \PP{B} = \boxed{0.8}$.

\subproblema{}
Since they are mutually exclusive, this is just the probability of $A$ occuring, which is $\boxed{0.3}$.

\subproblema{}
Since the events are mutually exclusive, they cannot both happen at once.
Hence, the $\PP{A \cap B} = 0$.


\setcounter{problem}{10}
\subproblema{}
The probability that a student wears either is the complement of the probability that they wear neither, which is $1 - 0.6 = \boxed{0.4}$.


\subproblema{}

Let $\PP{r}$ be the probability that a student wears a ring, and let $\PP{n}$ be the probability that a student wears a necklace.
\begin{align*}
  \PP{r \cup n} &= \PP{r} + \PP{n} - \PP{r \cap n} \\
  \PP{r \cap n} &= \PP{r} + \PP{n} - \PP{r \cup n} \\
                      &= 0.2 + 0.3 - 0.4 \\
                      &= \boxed{0.1} \\
\end{align*}

\setcounter{problem}{22}
\problem{}

Suppose the first die lands on $6$.
The second die cannot possibly have a higher value, so this case contributes $0$.

If the first die lands on $5$, then the second die must land on $5$ in order to have a higher value.
This case contributes $\frac{1}{36}$.

We can continue, and see that if the first die lands on $4$, $3$, $2$, or $1$, the probabilities that each case contributes is $\frac{2}{36}$, $\frac{3}{36}$, $\frac{4}{36}$, and $\frac{5}{36}$, respectively.

The sum of the probability contributions is $\frac{15}{36} = \boxed{\frac{5}{12}}$.


\end{document}