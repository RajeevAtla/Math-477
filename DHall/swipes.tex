\documentclass{article}
\usepackage[stdmargin, noindent]{../rajeev}

\begin{document}

\begin{center}
  \Large \textbf{How To Get More Swipes For Your Friends}
\end{center}

\begin{abstract}
  It's really unfortunate that Rutgers limits their meal plans to contain only 10 guest swipes per semester,
  especially for those who are limited to meal plans and want to eat with their friends,
  not all of who have meal plans.
  Using the fact that most people who purchase a meal plan don't use the entirety of their meal swipes each semester,
  we devise a plan that allows you to spend more than 10 of your swipes on non-meal plan using people each semester.
\end{abstract}

\section{Introduction}
Rutgers only lets you swipe in $10$ of your friends each semester.
This is problematic for those who are limited to the meal plan and can't really afford to eat out that often,
but also want to eat with their friends.
The common approach seems to be to use these guest swipes conservatively,
only using them on close friends and only doing so scarcely.
However,
as the end of the semester approaches,
these guest swipes grow more and more valuable,
as many have gone through the entirety of their meager $10$ guest swipes.
In the past,
many students have unsuccesfully petitioned Rutgers Dining to increase the limit on the amount of guest swipes they can use each semester,
only to be batted aside.

\section{Solution}
The key point in the solution we propose is the fact that many don't actually finish their meal swipes for the semester.
We instead propose using the actual swipes instead of the guest swipes,
which can be used more efficiently later.
Suppose $n$ people enter the dining hall,
of whom $a \in \NN$ have meal swipes and $b \in \NN$ don't.
By definition,
$a + b = n$.

\subsection{Case 1: $a = b$}
All $a$ people first swipe themselves in,
and find themselves a table.
The $b$ people remain outside,
where they attempt to be inconspiciuous.
One person leaves the dining hall,
carrying $a-1$ extra IDs for the $b$ people outside.
There is $1$ person left outside.
Another person from the $a$ group leaves,
taking another ID from the $a$ group with him.
They give the last person the ID that 

\end{document}