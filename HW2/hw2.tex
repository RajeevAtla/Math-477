\documentclass{article}
\usepackage[stdmargin, noindent]{../rajeev}

\begin{document}
\setcounter{problem}{2} 
\problem{}

$EF$ is the event that the sum of the first roll is odd and in second roll, at least one die lands on 1.
$E \cup F$ is the event that the sum of a roll is odd and one die lands on 1.
$FG$ is the event that the first roll has at least 1 die that lands on 1 and the second roll sums to 5.
$EF^c$ is the event that the first roll sums to an odd number and the none of the dice in the second roll land on 1.
$EFG$ is the event that the first roll sums to an odd number, the second roll has at least one die that lands on 1, and the third roll sums to 5.

\setcounter{problem}{8}

\subproblema{}
The events are mutually exclusive, so we have $\PP{A \cup B} = \PP{A} + \PP{B} = \boxed{0.8}$.

\subproblema{}
Since they are mutually exclusive, this is just the probability of $A$ occuring, which is $\boxed{0.3}$.

\subproblema{}
Since the events are mutually exclusive, they cannot both happen at once.
Hence, the $\PP{A \cap B} = 0$.


\setcounter{problem}{10}
\subproblema{}
The probability that a student wears either is the complement of the probability that they wear neither, which is $1 - 0.6 = \boxed{0.4}$.


\subproblema{}

Let $\PP{r}$ be the probability that a student wears a ring, and let $\PP{n}$ be the probability that a student wears a necklace.
\begin{align*}
  \PP{r \cup n} &= \PP{r} + \PP{n} - \PP{r \cap n} \\
  \PP{r \cap n} &= \PP{r} + \PP{n} - \PP{r \cup n} \\
                      &= 0.2 + 0.3 - 0.4 \\
                      &= \boxed{0.1} \\
\end{align*}

\setcounter{problem}{22}
\problem{}

You must go 4 steps to the right and 3 steps up to get to point $B$.
The multiple paths are created by the way these steps are arranged.
This is basically the same problem as the number of ways to arrange the letters in the string RRRRUUU, where R corresponds to a movement to the right and U corresponds to a movement up.
To account for the repeated, letters we divide the factorial of the total number of letters by the factorial of the number of times a letter is repeated, for each letter.
This means there are $\frac{7!}{4! 3!} = \boxed{35}$.


\end{document}