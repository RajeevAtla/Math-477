\documentclass{article}
\usepackage[stdmargin, noindent]{../rajeev}

\begin{document}

\begin{center}
  \Large HW 6
\end{center}

\problem{}

Let $X$ be a random variable denoting the winnings.
Since doing otherwise would be cumbersome,
we make a table illustrating what the possible values of $X$ are,
as well as what ball combinations they correspond to.

\begin{center}
  \begin{tabular}{ |c|c|c| } 
    \hline
    \textbf{Value of $X$} & \textbf{Ball 1} & \textbf{Ball 2} \\ 
    \hline
    4 & Black & Black \\
    2 & Black & Orange \\
    1 & Black & White \\
    0 & Orange & Orange \\
    -1 & White & Orange \\
    -2 & White & White \\
    \hline
  \end{tabular}
\end{center}

Note that \textbf{Ball 1} and \textbf{Ball 2} may be freely switched.

This means,
\begin{align*}
  \PP{X = 4} &= \frac{4}{14} \cdot \frac{3}{13} \\
             &= \boxed{\frac{6}{91}} \\
  \PP{X = 2} &= 2 \cdot \frac{4}{14} \cdot \frac{2}{13} \\
             &= \boxed{\frac{8}{91}} \\
  \PP{X = 1} &= 2 \cdot \frac{4}{14} \cdot \frac{8}{13} \\
             &= \boxed{\frac{32}{91}} \\
  \PP{X = 0} &= \frac{2}{14} \cdot \frac{1}{13} \\
             &= \boxed{\frac{1}{91}} \\
  \PP{X = -1} &= 2 \cdot \frac{8}{14} \cdot \frac{2}{13} \\
             &= \boxed{\frac{16}{91}} \\
  \PP{X = -2} &= \frac{8}{14} \frac{7}{13} \\
             &= \boxed{\frac{4}{13}} \\
\end{align*}

\setcounter{problem}{5}
\problem{}
There are $4$ possible values of $X$: $\set{3, 1, -1, -3}$.
The first and last values correspond to all heads and all tails,
respectively.
The second and third values correspond to $2$ heads and $2$ tails,
respectively.

\begin{align*}
  \PP{X = 3} &= \pars{\frac{1}{2}}^3 \\
             &= \boxed{\frac{1}{8}} \\
  \PP{X = 1} &= \binom{3}{2} \pars{\frac{1}{2}}^2 \pars{\frac{1}{2}} \\
             &= \boxed{\frac{3}{8}} \\
  \PP{X = -1} &= \binom{3}{2} \pars{\frac{1}{2}}^2 \pars{\frac{1}{2}} \\
             &= \boxed{\frac{3}{8}} \\
   \PP{X = -3} &= \pars{\frac{1}{2}}^3 \\
             &= \boxed{\frac{1}{8}} \\
\end{align*}


\setcounter{problem}{20}
\problem{}
\subproblema{}

$\EE{X}$ will be greater than $\EE{Y}$,
as there are more buses than there are students.

\subproblema{}
\begin{align*}
  \EE{X} &= 40 \cdot \frac{40}{148} + 33 \cdot \frac{33}{148} + 25 \cdot \frac{25}{148} + 50 \cdot \frac{50}{148} \\
         &= \frac{1}{148} \pars{1600 + 1089 + 625 + 2500} \\
         &= \frac{5814}{148} \\
         &= \boxed{\frac{2907}{74}} \\
         & \approx 39.284 \\
  \EE{X} &= 40 \cdot \frac{1}{4} + 33 \cdot \frac{1}{4} + 25 \cdot \frac{1}{4} + 50 \cdot \frac{1}{4} \\
         &= \frac{148}{4} \\
         &= \boxed{37} \\
\end{align*}

\setcounter{problem}{24}
\problem{}
\subproblema{}
When $X=1$, only one of the coins may be heads.
Therefore,
\begin{align*}
  \PP{X = 1} &= \frac{3}{5} \cdot \frac{3}{10} + \frac{2}{5} \cdot \frac{7}{10} \\
             &= \boxed{\frac{23}{50}} \\
\end{align*}

\subproblema{}
We must first calculate $\PP{X = 0}$ and $\PP{X = 2}$.
\begin{align*}
  \PP{X = 0} &= \frac{2}{5} \cdot \frac{3}{10} \\
             &= \frac{3}{25} \\
  \PP{X = 2} &= \frac{3}{5} \cdot \frac{7}{10} \\
             &= \frac{21}{50} \\
\end{align*}

Now, we can calculate $\EE{X}$.
\begin{align*}
  \EE{X} &= 0 \cdot \frac{3}{25} + 1 \cdot \frac{23}{50} + 2 \cdot \frac{21}{50} \\
         &= \frac{23}{50} + \frac{42}{50} \\
         &= \boxed{\frac{13}{10}} \\
         &= 1.3 \\
\end{align*}


\end{document}